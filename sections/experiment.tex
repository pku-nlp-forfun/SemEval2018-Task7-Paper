\section{Experiment}
\label{sec:experiment}

In the previous section, we mentioned many experiments on the different corpus, embedding tools and model. We will list the performance of each attempt and highlight the best performance on both tasks.

Because the TOPIC class in subtask 1.1 is unbalanced. So we can see the phenomenon that subtask 1.2 will overall be higher than subtask 1.1.

\subsection{Different Corpus and Embedding Model}
\label{sec:different_corpus_and_embedding_model}

Word embedding almost the important thing in NLP task. So we take some word embedding method and train dataSet to tune the effect of word embedding.

First of all, the train dataSet of competition is so small that the effect of training word embedding is bad. So we import some outer dataSet to optimization the effect of word embedding. The domain of our task is about scientific papers. So, we load dataset on Citation Network Dataset. We test dblp v5, dblp v10, and acm v9, we find dblp v10 have best f1 score both in trainSet and testSet.

We also do some work on different word embedding methods, like word2vec, Bert, fastText. FatsText do the best job in our experiment. Bert doesn't have a good effect on our task. We think it may be caused by the embedding size of the pre-train model.

\begin{table}[htbp!] % here top bottom page (! = remove further restrictions)
    \begin{tabular}{llcccc}
    \toprule
        \multicolumn{2}{c}{Task}                                                 & \multicolumn{2}{c}{Subtask 1.1} & \multicolumn{2}{c}{Subtask 1.2} \\
    \midrule
        Corpus & Embedding Model                                                               & Macro-F1        & Micro-F1       & Macro-F1        & Micro-F1       \\
    \midrule
        Pre-trained DBLP v5 & word2vec    & 88.99            & 78.99           & 64.99            & 38.99           \\
        ACM v9 + DBLP v10   & word2vec & 87.99            & 78.99           & 64.99            & 38.99           \\
        DBLP v5   & word2vec & 87.99            & 78.99           & 64.99            & 38.99           \\
        DBLP v10   & word2vec & 87.99            & 78.99           & 64.99            & 38.99           \\
        ACM v9 + DBLP v10   & fastText & 87.99            & 78.99           & 64.99            & 38.99           \\
        DBLP v10   & fastText & 87.99            & 78.99           & 64.99            & 38.99           \\
        ACM v9 + DBLP v10   & BERT & 87.99            & 78.99           & 64.99            & 38.99           \\
    \bottomrule
    \end{tabular}
\caption{Comparison between different corpus using LibLinear logistic regression model (in \%)}
\label{tab:corpus}
\end{table}


\subsection{Different Feature}
\label{sec:different_feature}

We also do some work on feature engineering. We test the effect of different '[PAD]' position. Clustering is also one way to improve our model. We add both one-hot and word embedding of middle words between two entities to feature lists. And we also do some artificial features like have including '\_' in middle words between two entities. We do some experiment which one by one add features to model.

\begin{table}[htbp!] % here top bottom page (! = remove further restrictions)
    \centering
    \begin{tabular}{lccccc}
    \toprule
        \multicolumn{2}{c}{Task}                            & \multicolumn{2}{c}{Subtask 1.1 Test}  & \multicolumn{2}{c}{Subtask 1.1 Training} \\
    \midrule
        Model & Feature                                     & Macro-F1         & Micro-F1           & Macro-F1         & Micro-F1       \\
    \midrule
        LibLinear\\ Logistic Regression    & embedding only & \bf52.01         & -                  & 49.97            & -              \\
        LibLinear\\ Logistic Regression    & + one-hot      & 51.43            & -                  & 49.25            & -              \\
        LibLinear\\ Logistic Regression    & + shape        & 50.61            & -                  & 49.67            & -              \\
        LibLinear\\ Logistic Regression    & + cluster      & 51.55            & -                  & 49.31            & -              \\
        LibLinear\\ Logistic Regression    & + e1, e2       & 49.93            & -                  & 49.97            & -              \\
        Scikit Learn\\ Logistic Regression & embedding only & 50.37            & -                  & 52.12            & -              \\
        Scikit Learn\\ Logistic Regression & + one-hot      & 51.10            & -                  & 52.37            & -              \\
        Scikit Learn\\ Logistic Regression & + shape        & 51.29            & -                  & 51.81            & -              \\
        Scikit Learn\\ Logistic Regression & + cluster      & 51.90            & -                  & 51.97            & -              \\
        Scikit Learn\\ Logistic Regression & + e1, e2       & 50.92            & -                  & \bf52.99         & -              \\
    \bottomrule
    \end{tabular}
\caption{Comparison between differen feature based on LibLinear and Scikit Learn logistic regression model (in \%)}
\label{tab:feature}
\end{table}


\subsection{Different Model}
\label{sec:different_model}

Except for linear regression model, we also do some experiment on SVM, LinerSVC,DecisionTreeClassifier , TextCNN.

\begin{enumerate}
    \item Linear Regression use L2-regularized logistic regression type, cost=0.05, epsilon=0.1
    \item SVM use L2-penalty, squared hinge as loss function, C=1.0, OVR on multi-classification
    \item Decision Tree use gini impurity as criterion, max depth=$\infty$, min sample split=2
    \item TextCNN use 128 filter, filter size=[6,7,8], embedding size=300, learning rate = 0.0003, batch size = 64, decay step=1000, train around 300epochs, we get a local optimum train f1 score.
\end{enumerate}

\begin{table}[htbp!] % here top bottom page (! = remove further restrictions)
    \centering
    \begin{tabular}{lcccc}
    \toprule
        \multicolumn{1}{c}{Task}           & \multicolumn{2}{c}{Subtask 1.1}    & \multicolumn{2}{c}{Subtask 1.2}    \\
    \midrule
        Model Name                         & Macro-F1         & Micro-F1        & Macro-F1         & Micro-F1        \\
    \midrule
        LibLinear\\ Logistic Regression    & 52.01            & -               & 69.15            & \bf81.41        \\
        Scikit Learn\\ Logistic Regression & \bf52.77         & -               & 75.06            & 76.04           \\
        Scikit Learn\\ Linear SVM          & 51.45            & -               & 70.29            & 75.56           \\
        Scikit Learn\\ Decision Tree       & 36.65            & -               & -                & -               \\
        TextCNN                            & 51.20            & 65.50           & \bf77.35         & 80.53           \\
    \bottomrule
    \end{tabular}
\caption{Comparison between differen machine learning model using fastText embedding model on DBLP v10 corpus (in \%)}
\label{tab:model}
\end{table}


\subsection{Imbalance Data}
\label{sec:impalance_data}

The train dataset is an obvious imbalance set in subtask 1.1 Topic class. So we over-sampling the train set. We, in turn, add subtask 1.2 Train Topic, subtask 1.2 Test topic, subtask 1.2 Train, subtask 1.2 Test, subtask 1.2 to subtask1.1. And test the effect of over-sampling.

\begin{table}[htbp!] % here top bottom page (! = remove further restrictions)
    \begin{tabular}{lccccc}
    \toprule
        \multicolumn{1}{c}{Training Data}            & \multicolumn{2}{c}{Subtask 1.1 Test}  & \multicolumn{2}{c}{Subtask 1.1 Training} \\
    \midrule
        Training Data                                & Macro-F1         & Micro-F1           & Macro-F1         & Micro-F1       \\
    \midrule
        only subtask 1.1 training data               & 50.92            & 65.17              & 52.99            & -              \\
        + subtask 1.2 TOPIC data                     & 50.20            & 64.41              & 52.62            & -              \\
        + subtask 1.2 and test set TOPIC data        & 50.50            & 68.12              & 46.22            & -              \\
        + all the subtask 1.2 data                   & \bf53.73         & \bf71.07           & \bf67.20         & -              \\
        + all the test data                          & 52.82            & 68.26              & 66.73            & -              \\
        + everything                                 & 52.82            & 70.22              & 66.42            & -              \\
    \bottomrule
    \end{tabular}
\caption{Comparison between using different training data on Scikit Learn logistic regression model in order to solve the data imbalance problem on subtask 1.1 (in \%)}
\label{tab:data_imbalance}
\end{table}


The evaluation shows that class imbalance impacts the effect of subTask 1.1. But more data don't mean a better effect. When we add all data of subTask 1.2 including task and train data, we found the effect is worse than data fusing only subTask1.2 train data. And the over-sampling Topic class is not significant.

\subsection{Pad Position}
\label{sec:pad_position}

The lens between two entities is different in dataSet. So we should align word list before all word. We can put '[PAD]' before the entity1, after entity1, before entity2, after entity2. The result of 4 methods should be different. So we take some experimentation to explore this problem. We use TextCNN with Task1.1+Task1.2 Train dataSet, use 128 filters, filter size=[6,7,8], embedding size=300, learning rate = 0.0003, batch size = 64, decay step=1000.

\begin{table}[htbp!] % here top bottom page (! = remove further restrictions)
    \begin{tabular}{lccccc}
    \toprule
        \multicolumn{1}{c}{Training Data}          & \multicolumn{2}{c}{Subtask 1.1 Test}  & \multicolumn{2}{c}{Subtask 1.2 Test} \\
    \midrule
        Pad Position                               & Macro-F1         & Micro-F1           & Macro-F1         & Micro-F1       \\
    \midrule
        pad before entity1                         & 59.74            & 66.95              & 71.22            & 77.29         \\
        pad after entity1                          & 56.79            & 66.67              & 72.76            & 80.79         \\
        pad before entity2                         & \bf67.74         & \bf72.03           & \bf73.03         & \bf81.01      \\
        pad after entity2                          & 57.70            & 65.63              & 72.48            & 80.23         \\
    \bottomrule
    \end{tabular}
\caption{Comparison between padding different position on TextCNN model and use subtask 1.1 + subtask 1.2 training data to show the effect (in \%)}
\label{tab:data_pad}
\end{table}


The evaluation shows that padding position is a vital parameter in our model. In all subTask, we found that adding pad before entity1 is a good way to improve the performance of our model.